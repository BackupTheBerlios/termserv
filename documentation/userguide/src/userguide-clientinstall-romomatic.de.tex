\section{Erstellen eines Etherbootimages mit der ROM-o-matic}

Marty Connor betreibt auf \texttt{http://rom-o-matic.net} einen Service,
bei dem speziell konfigurierte Etherboot-Images dynamisch erzeugt und
heruntergeladen werden k�nnen. Die Benutzung ist einfach:

Auf der Startseite w�hlt man die gew�nschte Version von Etherboot.
Normalerweise sollte man hier die neueste "Production Release" w�hlen.
Man gelangt dann zu einem Formular mit 4 Punkten. Mit dem ersten Punkt
w�hlt man den Typ seiner Netzwerkkarte. Wenn nicht klar ist, welcher der
richtige ist, hilft ein Blick auf die Etherboot-Homepage
\texttt{http://etherboot.sourceforge.net} bzw. die Netzwerkkartendatenbank
auf \texttt{http://etherboot.sourceforge.net/db}
Unter dem zweiten Punkt kann man verschiedene Optionen einstellen, das ist
normalerweise nicht n�tig. Mit dem 3. Knopf w�hlt man das gew�nschte Format
(welches ben�tigt wird, h�ngt von der Startmethode ab, und wird in den
zugeh�rigen Kapiteln beschrieben), und der 4. schlie�lich generiert das Image.
