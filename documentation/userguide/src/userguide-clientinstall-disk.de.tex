\section{Terminals mit Festplatte}

Ist eine Festplatte in eimem Terminal �berhaupt sinnvoll, da ja lokal sowieso
nichts gespeichert wird?
Ja, wenn nur wenig Hauptspeicher zur Verf�gung steht, kann man sie sehr gut
als virtuellen Arbeitsspeicher gebrauchen.
In allen anderen F�llen sollten evtl. vorhandene Festplatten eher ausgebaut
werden, da sie unn�tig Strom verbrauchen, L�rm machen und aufgrund mechanisch
bewegter Teile irgendwann kaputt gehen.

Das Format des Images bei der ROM-o-matic ist dasselbe wie beim Starten von
Diskette, nur das �berspielen des Images auf die Festplatte funktioniert
naturgem�� etwas anders.

Hier gibt es zwei Wege: �ber Diskette oder �ber Einbau der Festplatte in einen
anderen PC.

Wenn das Terminal �ber ein Diskettenlaufwerk verf�gt, kopiert man
die Imagedatei auf eine formatierte Diskette, unter Linux t. B. mit
\begin{verbatim}
cp eb-5.0.2-ne.lzdsk /floppy/
\end{verbatim}
Hierbei ist angenommen, da� die Diskette unter \texttt{/floppy/} gemountet ist.
Nat�rlich kann das Kopieren auch auf einem DOS/Windows-Rechner geschehen.

Nun kann man das Terminal mit einem Taschenlinux (die Rettungsdiskette
g�ngiger Linuxdistributionen geht auch) starten.
Nachdem man sich als root angemeldet hat bzw. auf andere
Weise eine shell mit root-Rechten ge�ffnet hat, legt man die zuvor erstellte
Diskette ein und mountet sie:
\begin{verbatim}
mount /dev/fd0 /mnt
\end{verbatim}
Dazu mu� das Verzeichnis /mnt existieren. Falls nicht, legt man es mit dem
Befehl
\begin{verbatim}
mkdir /mnt
\end{verbatim}
an. Nun kopiert man das Etherboot-Image auf die Festplatte:
\begin{verbatim}
cat /mnt/eb-5.0.2-ne.lzdsk > /dev/hda
\end{verbatim}
Falls es sich um eine SCSI-Festplatte handelt, mu� statt \texttt{/dev/hda}
entsprechend \texttt{/dev/sda} eingegeben werden.

Besitzt das Terminal kein Diskettenlaufwerk ist es etwas aufwendiger: Man baut
die Festplatte aus und in einen Linux-PC ein.
Wenn die Platte z. B. als Master am zweiten IDE-Kanal h�ngt, wird mit 
\begin{verbatim}
cat /mnt/eb-5.0.2-ne.lzdsk > /dev/hdc
\end{verbatim}
das Image kopiert. Auch hier mu� man wieder aufpassen: Ein Tippfehler bei
\texttt{/dev/hdc} kann sehr unangenehme Folgen haben!

Nun kann die Platte wieder in das Terminal eingebaut werden.
