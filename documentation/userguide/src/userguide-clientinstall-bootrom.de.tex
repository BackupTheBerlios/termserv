\section{Terminals mit Booteprom}

Das Starten eines Terminals �ber ein Booteprom ist die eleganteste Methode,
die auch bei entsprechenden kommerziell vertriebenen L�sungen angewandt wird.
Der wesentliche Vorteil ist, da� man ohne bewegte Teile auskommt. Der Nachteil
ist, da� fertig programmierte Booteproms bzw. Netzwerkkarten mit fertigem
Booteprom recht teuer sind, und da� man zum Selberbrennen ein
Epromprogrammierger�t und eine Eproml�schlampe braucht (sofern es sich nicht
um Flasheproms handelt, die k�nnen ohne zus�tzliche Ger�te direkt
``in circuit'' programmiert werden).

Das Format des Images bei der ROM-o-matic ist ``Binary ROM Image (.lzrom)''.

Hat man das Gl�ck und eine Netzwerkkarte mit Flasheprom, so ist die
erhaltene Datei mit dem zur Netzwerkkarte geh�rigen
Flashprogramm in das Flasheprom zu laden.

Hat man eine Karte mit einem Sockel f�r ein normales EPROM, so ist die
erhaltene Datei mit dem Epromprogrammierger�t in das Eprom zu brennen.
Anschlie�end mu� das Eprom in den Sockel eingesetzt werden, und zwar so,
da� die Aussparung an der einen Stirnseite mit der entsprechenden Markierung
des Sockels �bereinstimmt.

Die meisten Etherbootimages sind 16 kB gro�. Die passenden Eproms haben die
Bezeichnung 27c128. Da jedoch die 32 kB-Eproms 27c256 pinkompatibel zum 27c128
und wesentlich h�ufiger verkauft werden und somit billiger sind, empfiehlt
sich dieser Typ.
Damit das auch in Karten funktioniert, die eigentlich nur 16 kB-Typen
vertragen, brennt man nicht das Image direkt, sondern ein 32 kB-Image
(hier \texttt{eb-5.0.2-ne-big.lzrom}), das man sich z. B. mit
\begin{verbatim}
cat eb-5.0.2-ne.lzrom eb-5.0.2-ne.lzrom > eb-5.0.2-ne-big.lzrom
\end{verbatim}
unter Linux oder
\begin{verbatim}
copy eb-5.0.2-ne.lzrom eb-5.0.2-ne.lzrom eb-5.0.2-ne-big.lzrom
\end{verbatim}
unter DOS/Windows erzeugt.

Meistens mu� das Eprom/Flasheprom noch mit einem Konfigurationsprogramm
aktiviert werden, bevor es benutzt werden kann.

Dieses Kapitel enth�lt nur die n�tigsten Punkte und ist eher als �berblick
�ber die M�glichkeiten gedacht, die man mit einem Booteprom hat. F�r
ausf�hrlichere Informationen, z. B. auch wie man Etherboot bei neueren PCs
direkt ins BIOS integrieren kann, oder wie ein vorhandenes PXE-kompatibles
Booteprom benutzt werden kann, sei auf die Etherboot-Dokumentation verwiesen.

