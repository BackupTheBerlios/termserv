\section{Idee und Prinzip}
Die Idee des Terminalservers ist simpel: Warum Software und 
Rechenkapazit�t auf jedem Endger�t vorhalten, dass z.B. beim 
Einsatz einer Textverarbeitung die allermeiste Zeit nur 
``D�umchen dreht''? Warum nicht die Kapazit�t eines Rechners ausnutzen, 
indem man mehrere Benutzer gleichzeitig an ihm arbeiten l�sst?
 
Da nat�rlich sich nicht mehrere Benutzer gleichzeitig vor einen 
Bildschirm und eine Tastatur setzen und arbeiten k�nnen, verteilt 
der Rechner seine Kapazit�t �ber ein Netz an mehrere ``dumme'' 
Terminals, die im Grunde nichts weiter darstellen als ein 
Bildschirm, eine Tastatur und ein Endger�t, das ohne Laufwerke 
und gro�en Speicherausausbau auskommt, das Terminal eben.
 
Prinzipiell sind zwei Formen des Terminals im Einsatz:
\begin{itemize}
\item X-Terminal \\
Dies sind z.B. schwachbr�stige Altrechner mit kleinen Festplatten, 
die nur ein Minimalsystem aufgespielt haben und sich alle Anwendungen vom Server holen.
\item Diskless-Client \\
Diese Rechner enthalten keine Laufwerke, alle erfolderlichen 
Programme holen sich die Ger�te �ber die Netzwerkkarte mit Eprom.
\end{itemize}
Eine Terminalserverl�sung besteht also aus einem gut ausgebauten Server, 
auf dem alle Software installiert und gewartet wird und den mit ihm 
verbundenen, so gut wie wartungsfreien Terminals. Wartungsfrei deswegen, 
weil auf ihnen keine Software installiert ist und keine Festplatten kaputt gehen k�nnen.

