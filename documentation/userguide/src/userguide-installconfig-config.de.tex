\section{Konfiguration}
�ber webmin
 
\subsection{Anpassungen an ein vorhandenes Netzwerk}
�ber Webmin
 
\subsection{Einrichtung des DHCP-Servers}
Der DHCP-Dienst kann entweder auf dem Terminalserver oder auf einem anderen
Rechner laufen. Einzige Bedingung ist, da� der andere Rechner von den Terminals
�ber das Netzwerk erreicht werden kann.

Befinden sich mehr als ein DHCP-Server im Netz, so ist unbedingt darauf zu
achten, da� sich die IP-Adre�bereiche, die jeder bedient, nicht �berschneiden.
Anderenfalls h�ngt es vom Zufall ab, von welchem DHCP-Server eine Anfrage
nach einer IP-Adresse zuerst beantwortet wird.

\subsubsection{DHCP auf dem Terminalserver}
�ber Webmin
 
\subsubsection{DHCP auf Arktur}
Der c't/ODS Kommunikationsserver Arktur enth�lt bereits einen DHCP-Server.
Wenn sich au�er den Terminals noch andere Rechner im Netz befinden
und den DHCP-Server benutzen, ist es empfehlenswert, diesen auch f�r die
Terminals mitzubenutzen. Dazu sind folgende �nderungen in der Datei
/\texttt{etc/dhcpd.conf} auf Arktur n�tig:

\begin{enumerate}
\item Eine Gruppe
\begin{verbatim}
group {
    option next-server termserv;
}
\end{verbatim}
ist zu erstellen. \texttt{termserv} ist hierbei der Name des Terminalservers.
Hiermit wird allen Terminals, die sich innerhalb dieser Gruppe befinden,
mitgeteilt, wie der n�chste Server (von dem der Linuxkernel geholt wird) hei�t.
Werden mehrere Terminalserver benutzt, bekommt jeder seine eigene Gruppe.
\item Pro Terminal ist innerhalb dieser Gruppe ein Eintrag der Form
\begin{verbatim}
    host Client-A40 {
        hardware ethernet 08:00:07:26:c0:a5;
        filename "/tftpboot/lts/vmlinuz.ne2000";
        fixed-address Client-A40;
    }
\end{verbatim}
zu erstellen.

\texttt{Client-A40} ist der Name des entsprechenden Terminals. Der
Nameserver mu� diesen Namen kennen und auf eine IP-Adresse aufl�sen. Eine
Tabelle der auf Arktur voreingerichteten Adre�bereiche findet sich auf Seite
\pageref{dns-arktur}.

\texttt{08:00:07:26:c0:a5} ist die MAC-Adresse der Netzwerkkarte.

\texttt{/tftpboot/lts/vmlinuz.ne2000} ist der Name des Kernels, der gebootet
werden soll.
Er mu� sich auf dem oben angegebenen Rechner (hier \texttt{termserv})
befinden, und nat�rlich zur verwendeten Netzwerkkarte passen. Je nach
verwendeter Linuxdistribution kann es auch n�tig sein, das f�hrende
\texttt{/tftpboot} wegzulassen. Im Zweifelsfall hilft ausprobieren.

\item Wenn die in Tabelle \ref{dns-arktur} angegebenen IP-Adre�bereiche
nicht ausreichen, ist analog zu den schon vorhandenen Subnetzeintr�gen
ein neues Subnetz zu definieren, z. B.
\begin{verbatim}
subnet 192.168.5.0 netmask 255.255.255.0 {
    option routers 192.168.5.1;
}
\end{verbatim}
\end{enumerate}

Die hier beschriebene Konfigurationsmethode stellt nur eine unter vielen
M�glichkeiten dar. Wer tiefer einsteigen m�chte, sei auf die Manualseite zu
\texttt{dhcpd.conf}\footnote{in Englisch, zu lesen mit dem Befehl
\texttt{man dhcpd.conf}} verwiesen.

\subsection{Einrichtung des DNS-Servers}

\subsubsection{DNS auf dem Terminalserver}
�ber Webmin
 
\subsubsection{DNS auf Arktur}
An der Konfiguration des standardm��ig auf Arktur laufenden Nameservers sind
normalerweise keine �nderungen n�tig. Die verf�gbaren Namen und IP-Adressen
stehen in Tabelle \ref{dns-arktur}.

\begin{table}[h]
\begin{tabular}{ll}
Name			&IP-Adresse\\
\texttt{Client-A40}	&\texttt{192.168.0.40}\\
\multicolumn{2}{c}{\vdots}\\
\texttt{Client-A250}	&\texttt{192.168.0.250}\\
\texttt{Client-B10}	&\texttt{192.168.1.10}\\
\multicolumn{2}{c}{\vdots}\\
\texttt{Client-B250}	&\texttt{192.168.1.250}\\
\texttt{Client-C10}	&\texttt{192.168.2.10}\\
\multicolumn{2}{c}{\vdots}\\
\texttt{Client-C250}	&\texttt{192.168.2.250}\\
\texttt{Client-D10}	&\texttt{192.168.3.10}\\
\multicolumn{2}{c}{\vdots}\\
\texttt{Client-D250}	&\texttt{192.168.3.250}\\
\end{tabular}
\caption{Voreingerichtete Namen und IP-Adressen auf Arktur}
\label{dns-arktur}
\end{table}

\subsection{Das LTSP-Modul - Einrichtung der Terminal-Hardware} 
