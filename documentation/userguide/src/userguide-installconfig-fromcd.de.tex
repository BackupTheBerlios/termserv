\section{Installation von CD} 
Die Pakete des Terminalservers liegen im rpm- und deb-Format vor. Das
deb-Format wird von der Debian-Distribution (und allen abgeleiteten wie
Progeny oder Corel) benutzt, das rpm-Format vom Rest.

\subsection{Immer ben�tigte Pakete}
Diejenigen Pakete, die auf jeden Fall ben�tigt werden, werden auf
Debian-Systemen durch Installieren des Pakets task-termserv automatisch
mitinstalliert, wenn die CD vorher dem Paketmanagementsystem mit
\begin{verbatim}
apt-cdrom add
\end{verbatim}
bekannt gemacht wurde (die CD mu� sich hierzu im Laufwerk befinden).

Auf rpm-basierten Systemen kann der Abh�ngigkeitsliste von task-termserv
entnommen werden, welche Pakete ben�tigt werden.

\subsection{Von der Terminalhardware abh�ngige Pakete}
Abh�ngig von den zur Verf�gung stehenden Terminals sind die Kernel- und
Xserverpakete.

\begin{table}[h]
\begin{tabular}{lll}
Xserver	&rpm-Paket			&deb-Paket\\
3dlabs	&lts\_x3dlabs-2.0-1.i386.rpm	&lts-xserver-3dlabs\_2.0-1\_all.deb\\
8514	&lts\_x8514-2.0-1.i386.rpm	&lts-xserver-8514\_2.0-1\_all.deb\\
agx	&lts\_xagx-2.0-1.i386.rpm	&lts-xserver-agx\_2.0-1\_all.deb\\
fbdev	&lts\_xfbdev-2.0-1.i386.rpm	&lts-xserver-fbdev\_2.0-1\_all.deb\\
i128	&lts\_xi128-2.0-1.i386.rpm	&lts-xserver-i128\_2.0-1\_all.deb\\
mach32	&lts\_xmach32-2.0-1.i386.rpm	&lts-xserver-mach32\_2.0-1\_all.deb\\
mach64	&lts\_xmach64-2.0-1.i386.rpm	&lts-xserver-mach64\_2.0-1\_all.deb\\
mach8	&lts\_xmach8-2.0-1.i386.rpm	&lts-xserver-mach8\_2.0-1\_all.deb\\
mono	&lts\_xmono-2.0-1.i386.rpm	&lts-xserver-mono\_2.0-1\_all.deb\\
p9000	&lts\_xp9000-2.0-1.i386.rpm	&lts-xserver-p9000\_2.0-1\_all.deb\\
s3	&lts\_xs3-2.0-1.i386.rpm	&lts-xserver-s3\_2.0-1\_all.deb\\
s3v	&lts\_xs3v-2.0-1.i386.rpm	&lts-xserver-s3v\_2.0-1\_all.deb\\
svga	&lts\_xsvga-2.0-1.i386.rpm	&lts-xserver-svga\_2.0-1\_all.deb\\
vga16	&lts\_xvga16-2.0-1.i386.rpm	&lts-xserver-vga16\_2.0-1\_all.deb\\
w32	&lts\_xw32-2.0-1.i386.rpm	&lts-xserver-w32\_2.0-1\_all.deb\\
\end{tabular}
\caption{Xserverpakete}
\end{table}

Welche Xserver ben�tigt werden, h�ngt von den Grafikkarten der Terminals ab.
Meistens wird der svga Xserver funktionieren.
Falls unklar ist, welcher Xserver n�tig ist, finden sich in der
Hardwaredatenbank von SuSE und auf \texttt{http://www.xfree86.org} weitere
Informationen.

\begin{table}[h]
\begin{tabular}{lll}
Netzwerkkarte	&rpm-Paket				&deb-Paket\\
3c509		&lts\_kernel\_3c509-2.2-0.i386.rpm	&lts-kernel-3c509\_2.2-1\_all.deb\\
3c905		&lts\_kernel\_3c905-2.2-0.i386.rpm	&lts-kernel-3c905\_2.2-1\_all.deb\\
all		&lts\_kernel\_all-2.2-0.i386.rpm	&lts-kernel-all\_2.2-1\_all.deb\\
eepro100	&lts\_kernel\_eepro100-2.2-0.i386.rpm	&lts-kernel-eepro100\_2.2-1\_all.deb\\
ne2000		&lts\_kernel\_ne2000-2.2-0.i386.rpm	&lts-kernel-ne2000\_2.2-1\_all.deb\\
rtl8139		&lts\_kernel\_rtl8139-2.2-0.i386.rpm	&lts-kernel-rtl8139\_2.2-1\_all.deb\\
tulip		&lts\_kernel\_tulip-2.2-0.i386.rpm	&lts-kernel-tulip\_2.2-1\_all.deb\\
\end{tabular}
\caption{Kernelpakete}
\end{table}

Die verf�gbaren Kernel decken nicht alle Netzwerkkarten ab. Der Kernel all
enth�lt alle Netzwerkkartentreiber. Wenn f�r die gew�snchte Karte
kein Kernel vorhanden ist, und der all-Kernel nicht funktioniert, mu� ein
Kernel selber erstellt werden und nach /tftpboot/lts/ kopiert werden. Wie das
geht, steht in der Etherboot-Dokumentation.

Tabellen evtl. in den Anhang?

