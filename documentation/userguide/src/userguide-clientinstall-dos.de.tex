\section{Terminals mit DOS}

Wenn auf dem Terminal als zweites Betriebssystem DOS l�uft, kann Etherboot
auch von dort aus gestartet werden. Mit dem Startmen� von MS-DOS 6.x oder
einer selbst geschriebenen Batch-Datei kann beim Starten zwischen dem
Betrieb unter DOS oder als Terminal gew�hlt werden.

Hier wird davon ausgegangen, da� DOS bereits auf der Fetsplatte installiert
ist. Das Format bei der ROM-o-matic ist ``DOS.COM Executable ROM Image''.
Die erhaltene Datei (z. B. eb-5.0.2-ne.com) kopiert man auf die Festplatte.
Beim Aufruf wird dann Etherboot gestartet - Fertig!
