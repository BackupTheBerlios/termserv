\section{Vorg�nge beim Starten eines Terminals}
 
\begin{itemize}
\item Das Terminal bootet. Nun wird ein Code auf dem Boot-ROM der Netzwerkkarte ausgef�hrt,
der alternativ auch von Diskette oder als DOS-Programm ausgef�hrt werden kann.
\item Der Code des Boot-ROMs versucht �ber das sogenannte BOOTP- oder DHCP-Protokoll einen Server
im Netzwerk zu finden, der dem Terminal anhand seiner Hardware-Ethernet-Adresse
eine IP-Adresse zuweisen und weiteren Code zur Ausf�hrung bereitstellen kann.
\item Der Server, i.d.R. der gleiche Rechner wie der Terminalserver, stellt dem suchenden
Terminal seine IP-Adresse, die IP-Adresse eines Boot-Servers sowie den Namen eines vom Boot-Server
zu ladenden Startcodes zur Verf�gung.
\item Das Terminal l�dt den Startcode mit dem TFTP-Protokoll, einem vereinfachten FTP-Protokoll.
Dieser Startcode ist ein entsprechend modifizierter Linux-Kernel.
\item Der Linux-Kernel wird auf den Terminal geladen und ausgef�hrt. Nachdem er gestartet
ist, holt er sich sein gesamtes Dateisystem vom Terminalserver mit Hilfe des
NFS-Protokolls.
\item Innerhalb dieses Dateisystems liegen Startscripts. Sie werden ausgef�hrt und
starten die graphische Oberfl�che.
\item Die graphische Oberfl�che X11 sucht einen XDMCP-Server, das ist ein Rechner,
der in der Lage ist, die Graphikbefehle der laufenden Programme auf die Terminals
zu �bertragen.
\item Wenn ein solcher Server gefunden ist, verbindet sich der Terminal mit ihm. Ab
diesem Punkt arbeitet der Benutzer auf dem Server.
\end{itemize}     
