\chapter{Der Terminalserver}
Die Sitaution ist bekannt: Die Netzwerke an Schulen werden gr��er und 
komplexer, die Anspr�che der Kollegen an die neuen Medien steigen, 
Schulbuchverlage, Hard- und Softwarehersteller verhei�en goldene Zeiten.
 
Leider sieht die Wirklichkeit vor Ort nicht immer so rosig aus und 
vielerorts realisieren auch die Schultr�ger, dass die Kosten f�r 
Anschaffung und Wartung der Ger�te, der Bedarf an Administration von 
Hard- und Software st�ndig w�chst. Schulen mit 40, 50, 80 oder gar 
100 und mehr Rechnern sind keine Exoten mehr. Netze solcher Gr��enordnung  
erfordern leistungsf�hige Server, kaskadierte Netze und  -- vor allem -- 
kompetente, rund um die Uhr verf�gbare Systemadminstratoren,  die sich ihre
Arbeit in der Industrie teuer bezahlen lassen.
 
St�ndig steigender Wartungsaufwand und steigende Kosten lassen an der 
Schule nach anderen L�sungen als den bisher eingesetzten suchen. Es 
erweist sich mehr und mehr als Unding, die Anwendungssoftware auf 
jedem einzelnen Rechner zu installieren und zu warten. Zwar gibt es 
auch hierf�r L�sungen, die aber vor allem durch eins gl�nzen: sie 
sind teuer. Schon deswegen teuer, weil f�r jeden Arbeitsplatz ein eigener 
kompletter Rechner gestellt und nach wenigen Jahren ersetzt werden muss.
 
Eine L�sungsm�glichkeit in diesem Dilemma bietet der Terminalserver. Er 
ist keine Erfindung des 21. Jahrhunderts, sondern greift einen bekannten 
und bew�hrten L�sungsansatz auf. 
