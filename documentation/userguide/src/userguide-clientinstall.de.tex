\chapter{Installation der Terminals}

Bevor es an die Installation der Terminals geht, ein paar Grundlagen
zum Startvorgang eines PCs:

Direkt nach dem Einschalten wird das BIOS (Basic Input and Output System)
gestartet. Dieses f�hrt ein paar Tests durch und l�dt danach das
Startprogramm des Betriebssystems. Wo es dieses findet, kann man im
Setup-Programm des BIOS einstellen.
Dort finden sich normalerweise die Optionen Diskette, Festplatte, CDROM
und bei neueren Rechnern oft auch Netzwerk.

Normalerweise wird vom Startprogramm dann das Betriebssystem geladen,
evtl. kann man zwischen mehreren w�hlen. In unserem Fall hei�t dieses
Startprogramm etherboot oder netboot. Beide erf�llen den gleichen Zweck:
Statt das Betriebssystem von Platte oder Diskette zu laden, wird es �ber
das Netz von einem Server geholt.

Im Folgenden gehen wir von etherboot aus, da es deutlich flexibler als
netboot ist und aktiver weiterentwickelt wird. Ein Unterschied zwischen
etherboot und netboot ist, da� etherboot f�r jede Netzwerkkarte einen eigenen
Treiber braucht, w�hrend netboot eine Art Mini-DOS zur Verf�gung stellt,
soda� mit netboot alle Netzwerkkarten verwendet werden k�nnen, f�r die
es einen DOS-Packet-Treiber gibt. In der Regel sollte man also etherboot
verwenden. Wenn die Netzwerkkarte von etherboot nicht unterst�tzt wird,
kann man es mit netboot versuchen.

Von wo aus etherboot geladen wird, ist f�r den weiteren Startvorgang des
Terminals unerheblich. Es hat allein praktische Gr�nde. Wir beschr�nken
uns hier auf drei Methoden: Start von Diskette, von Festplatte und von
einem Booteprom, die in den n�chsten Kapiteln beschrieben werden.

Weiterhin gehen wir davon aus, da� Zugang zum Internet besteht und die
Etherboot-Images von \texttt{http://rom-o-matic.net} heruntergeladen werden
k�nnen.

