\section{Adressierungen}
 
Die Rechner in einem Netzwerk ben�tigen eindeutige Adressen. Dies ist vergleichbar
mit der Post, die ebenfalls eindeutige Adressen zum Bef�rdern ben�tigt.
 
\subsection{MAC-Adressen}
 
MAC-Adressen sind Netzwerk-Adressierungen auf der technisch untersten Ebene.
Jede Netzwerkkarte hat eine fest eingerichtete MAC-Adresse (gelegentlich auch
``Hardware-Ethernet-Adresse'' genannt). Sie wird durch den Hersteller festgelegt.
W�rden zwei Netzwerkkarten in einem Netzwerk die gleiche MAC-Adresse besitzen,
w�rde dies zu einem Konflikt f�hren. Jedoch haben alle Hersteller �bereink�nfte
�ber die Verteilung von MAC-Adressen getroffen.
 
 
\subsubsection*{Beispiel}
 
Eine MAC-Adresse besteht aus sechs zweistelligen Hexadezimalzahlen, die meist
durch Doppelpunkte getrennt sind, wie etwa \emph{00:0D:84:F6:3A:10}. 
\subsection{IP-Adressen}
 
IP-Adressen geben eine logische Struktur in einem Netzwerk vor. Sie sind durch
den Systembetreuer konfigurierbar.
 
 
\subsubsection*{Beispiel}
 
Eine IP-Adresse besteht aus vier zweistelligen Hexadezimalzahlen, die jedoch
fast immer der �bersichtlichkeit halber als Dezimalzahlen geschrieben werden.
Sie werden mit Punkten voneinander getrennt. In einem Schulnetzwerk k�nnte der
Server die IP-Adresse \emph{192.168.1.1} besitzen. Der Proxy-Server besitzt
die IP \emph{192.168.1.2}. Die Terminals bekommen \emph{192.168.1.10}, \emph{192.168.1.11},
\dots
 
 
\subsection{Rechnernamen und Dom�nen}
 
IP-Adressen sind zur Strukturierung ganz n�tzlich, jedoch k�nnen in einem IP-Netzwerk
rein mathematisch �ber 4 Milliarden Rechner beheimatet sein! Damit man sich
kein ``Telefonbuch'' f�r IP-Adressen anlegen muss und stattdessen menschenlesbare
Namen verwenden kann, wurde das Dom�nennamenssystem DNS (``Domain Name System'')
entwickelt. DNS-Server in einem Netzwerk halten eine Liste �ber IP-Adressen
und ihre Namen vor, die von Terminals abgefragt werden kann.       

Das Konzept ist dom�nenorientiert, d.h. es existiert eine Hierarchie: arktur.cg-bamberg.de
ist ein deutscher (\emph{de}) Rechner des Clavius-Gymnasiums Bamberg (\emph{cg-bamberg})
mit dem Namen \emph{arktur}. Das L�nderk�rzel hei�t Top-Level-Domain (kurz TLD),
davor kommt die Dom�ne, davor dann entweder Rechner oder weitere Dom�nen.
 

