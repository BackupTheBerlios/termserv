\section{Terminals mit Startdiskette}

Eine Startdiskette ist die einfachste Methode, um ein Terminal zu starten,
weil sie schnell auf einem anderen Rechner erstellt werden kann und einfach
nur eingelegt zu werden braucht. Daher wird diese Methode auch gerne zum
Testen verwendet. Ein Nachteil ist, da� ein Diskettenlaufwerk bewegliche
Teile besitzt und damit staubanf�llig ist und mechanisch altert. Ein weiterer
Nachteil ist, da� die Startdiskette verlorengehen kann (dem kann man abhelfen,
indem man das Laufwerk mit eingelegter Diskette so einbaut, da� es von au�en
nicht zug�nglich ist, dann kann man es aber nicht f�r andere Zwecke
verwenden).

Das Format bei der ROM-o-matic ist ``Floppy Bootable ROM Image (.lzdsk)''.
Die erhaltene Datei (z. B. eb-5.0.2-ne.lzdsk) schreibt man sich unter Linux mit

\begin{verbatim}
cat eb-5.0.2-ne.lzdsk > /dev/fd0
\end{verbatim}

auf die Diskette im 1. Laufwerk. Aber Achtung: Wenn vorher etwas auf der
Diskette gespeichert war, ist es jetzt �berschrieben, und wenn man sich bei
\texttt{/dev/fd0} vertippt, kann Schlimmes passieren (�berschreiben der
Festplatte etc.) Nat�rlich braucht man f�r diesen Befehl Schreibrechte
auf dem Diskettenlaufwerk, als Benutzer root sollte es auf alle F�lle
funktionieren.

Da die Datei sehr klein ist, ist es normal, wenn die LED des Laufwerks nur
kurz aufleuchtet, und man fast keine Ger�usche h�rt.

Das wars! Jetzt kann das Terminal mit dieser Diskette gestartet werden.
